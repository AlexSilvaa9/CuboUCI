\documentclass{article}
\usepackage{fullpage}

%load needed packages
\usepackage{graphicx}
\usepackage{array}
\usepackage{booktabs}
\usepackage[utf8]{inputenc}
\usepackage[T1]{fontenc}
\usepackage{url}
\usepackage[spanish]{babel} % Paquete para el idioma español
\usepackage{float}  % Necesario para [H]
\usepackage{listings}
\usepackage{xcolor}

\definecolor{codegreen}{HTML}{5AB2FF}
\definecolor{morado}{HTML}{AD88C6}
\definecolor{BG}{HTML}{EEEEEE}
\definecolor{azul}{HTML}{4D869C}
\definecolor{sqlblue}{HTML}{FF8C00} % Color para las palabras clave SQL

% Estilo para DDL
\lstdefinestyle{ddlstyle}{
	language=SQL,
	backgroundcolor=\color{BG},
	commentstyle=\color{codegreen},
	basicstyle=\ttfamily\small,
	keywordstyle=\color{azul},
	stringstyle=\color{morado},
	showstringspaces=false,
	breaklines=true,
	frame=shadowbox,
	numbers=left,
	numberstyle=\tiny\color{gray},
	captionpos=b,
}

% Estilo para SQL
\lstdefinestyle{sqlstyle}{
	language=SQL,
	backgroundcolor=\color{BG},
	commentstyle=\color{codegreen},
	basicstyle=\ttfamily\small,
	keywordstyle=\color{sqlblue}, % Color diferente para palabras clave SQL
	stringstyle=\color{morado},
	showstringspaces=false,
	breaklines=true,
	frame=shadowbox,
	numbers=left,
	numberstyle=\tiny\color{gray},
	captionpos=b,
}

\begin{document}
	
	
	
	% Portada
	\begin{titlepage}
		\centering
		\vspace*{3cm}
		
		% Título destacado
		{\Huge \textbf{Análisis de datos (MDX)}\\[0.5cm]}
		
		% Espacio y logotipo (si lo tienes, por ejemplo el logo de tu universidad)
		\vspace{2cm}
		\includegraphics[width=0.3\textwidth]{images/uma_logo.jpg}\\[1cm]
		
		% Nombre del autor
		{\LARGE \textbf{Alejandro Silva Rodríguez}\\[0.5cm]}
		{\LARGE \textbf{Marta Cuevas Rodríguez}\\[0.5cm]}
		{\large \textit{Almacenes De Datos}\\
			Universidad de Málaga\\
		}
		
		\vfill
		
		% Fecha en la parte inferior de la página
		{\large Diciembre 2024}
	\end{titlepage}
	
	% indice
	\tableofcontents
	
	\newpage
\section{Introducción}
\label{sec:introduccion}

En el contexto hospitalario actual, el análisis avanzado de datos se ha convertido en una herramienta indispensable para optimizar la toma de decisiones y mejorar la gestión de recursos en áreas críticas como las Unidades de Cuidados Intensivos (UCI). El análisis detallado del gasto en medicamentos, que representa una proporción significativa de los costos operativos, requiere técnicas especializadas que permitan explorar grandes volúmenes de datos desde múltiples perspectivas.\\

Tras la construcción de un almacén de datos orientado al análisis del gasto en medicamentos, el siguiente paso lógico es implementar estructuras que soporten consultas analíticas avanzadas. Los cubos multidimensionales permiten una visión integral de los datos, facilitando la identificación de patrones, tendencias y áreas críticas de gasto. Además, el uso de consultas MDX (Multidimensional Expressions) habilita a los usuarios para realizar análisis dinámicos y obtener insights clave de manera rápida y eficiente. Este trabajo se centra en la creación y explotación de un cubo multidimensional diseñado específicamente para analizar el gasto en medicamentos en pacientes ingresados en UCI en hospitales de EE.UU.

\section{Objetivos}
\label{sec:objetivos}

El objetivo principal de este trabajo es diseñar, implementar y explotar un cubo multidimensional para analizar el gasto en medicamentos en las UCI mediante el uso de consultas MDX. Este propósito se concreta en los siguientes objetivos específicos:

\begin{itemize}
	\item Diseñar y construir un cubo multidimensional que permita explorar de manera eficiente el gasto en medicamentos desde múltiples dimensiones, como tiempo, tipo de medicamento y características del paciente.
	\item Implementar consultas MDX que permitan realizar análisis detallados.

\end{itemize}

	\section{Creación del Cubo Multidimensional}
	  - Una sección que describa, a modo de tutorial, el proceso para la creación del cubo de vuestro proyecto. 

	\subsection{Creacion de proyecto}
	\subsection{Origen de datos}
	\subsection{Dimensiones}
	
	
	\section{Consultas MDX}
	- Una sección con las consultas en MDX y una captura con el resultado de cada una de ellas (la imagen capturada no tiene por qué mostrar todas las tuplas resultantes). 

\subsection{Consulta 1}

\textbf{Explicación:}  
Aquí se describe el propósito y el enfoque de la consulta. Explica qué información se busca obtener y cómo se construye la consulta MDX.

\textbf{Resultado:}  
Incluye una breve descripción de los resultados obtenidos y, si es necesario, una tabla o gráfico para visualizarlos.

\subsection{Consulta 2}

\textbf{Explicación:}  
Aquí se describe el propósito y el enfoque de la consulta. Explica qué información se busca obtener y cómo se construye la consulta MDX.

\textbf{Resultado:}  
Incluye una breve descripción de los resultados obtenidos y, si es necesario, una tabla o gráfico para visualizarlos.

% Repite el mismo esquema para las consultas restantes

\subsection{Consulta 3}

\textbf{Explicación:}  
Aquí se describe el propósito y el enfoque de la consulta. Explica qué información se busca obtener y cómo se construye la consulta MDX.

\textbf{Resultado:}  
Incluye una breve descripción de los resultados obtenidos y, si es necesario, una tabla o gráfico para visualizarlos.

\subsection{Consulta 4}

\textbf{Explicación:}  
Aquí se describe el propósito y el enfoque de la consulta. Explica qué información se busca obtener y cómo se construye la consulta MDX.

\textbf{Resultado:}  
Incluye una breve descripción de los resultados obtenidos y, si es necesario, una tabla o gráfico para visualizarlos.

\subsection{Consulta 5}

\textbf{Explicación:}  
Aquí se describe el propósito y el enfoque de la consulta. Explica qué información se busca obtener y cómo se construye la consulta MDX.

\textbf{Resultado:}  
Incluye una breve descripción de los resultados obtenidos y, si es necesario, una tabla o gráfico para visualizarlos.

\subsection{Consulta 6}

\textbf{Explicación:}  
Aquí se describe el propósito y el enfoque de la consulta. Explica qué información se busca obtener y cómo se construye la consulta MDX.

\textbf{Resultado:}  
Incluye una breve descripción de los resultados obtenidos y, si es necesario, una tabla o gráfico para visualizarlos.

\subsection{Consulta 7}

\textbf{Explicación:}  
Aquí se describe el propósito y el enfoque de la consulta. Explica qué información se busca obtener y cómo se construye la consulta MDX.

\textbf{Resultado:}  
Incluye una breve descripción de los resultados obtenidos y, si es necesario, una tabla o gráfico para visualizarlos.

\subsection{Consulta 8}

\textbf{Explicación:}  
Aquí se describe el propósito y el enfoque de la consulta. Explica qué información se busca obtener y cómo se construye la consulta MDX.

\textbf{Resultado:}  
Incluye una breve descripción de los resultados obtenidos y, si es necesario, una tabla o gráfico para visualizarlos.

\subsection{Consulta 9}

\textbf{Explicación:}  
Aquí se describe el propósito y el enfoque de la consulta. Explica qué información se busca obtener y cómo se construye la consulta MDX.

\textbf{Resultado:}  
Incluye una breve descripción de los resultados obtenidos y, si es necesario, una tabla o gráfico para visualizarlos.

\subsection{Consulta 10}

\textbf{Explicación:}  
Aquí se describe el propósito y el enfoque de la consulta. Explica qué información se busca obtener y cómo se construye la consulta MDX.

\textbf{Resultado:}  
Incluye una breve descripción de los resultados obtenidos y, si es necesario, una tabla o gráfico para visualizarlos.


\section{Tutorial ejecutar consultas}


	 - Una sección con las instrucciones detalladas para que un evaluador pueda ejecutar las consultas en su máquina
	\section{Dificultades Encontradas}
	\label{sec:dificultades_encontradas}
	  - Una sección "problemas encontrados" que explique con cierto detalle los problemas que se han encontrado durante la realización de la práctica. Se permite que la información incluida en esta sección se encuentre dividida o dispersada a lo largo del documento. 

	\section{Conclusión}
	\label{sec:conclusion}
	
	Gracias a la correcta implementación del proceso ETL, se ha logrado organizar el almacén de datos de manera eficiente, optimizando la consulta de información relevante para la toma de decisiones. Este proceso ha permitido integrar y transformar datos provenientes de \cite{eicu_crd}. Además, la estructura obtenida no solo asegura la calidad y consistencia de los datos, sino que también sienta las bases para futuras ampliaciones o análisis más complejos, promoviendo la escalabilidad y adaptabilidad del sistema.
	
	En conclusión, este proyecto ETL aporta un modelo sólido y adaptable para la gestión y análisis de datos en el ámbito hospitalario, contribuyendo a una administración más eficiente de los recursos y a una mejora potencial en la atención a los pacientes.
	
	\newpage
	\section{Acceso al Repositorio}
	
	Toda la información adicional, incluyendo el código fuente y la documentación completa de este proyecto, está disponible en el repositorio de GitHub \cite{silva2024github}.
	
	% Incluir la bibliografía
	\bibliographystyle{plain}  % Estilo de la bibliografía (por ejemplo, plain, alpha, ieee, etc.)
	\bibliography{bibli}  % Nombre del archivo .bib sin la extensión
	
\end{document}
